\documentclass[11pt,twoside,utf8,final]{fithesis} %twoside
\usepackage{amsmath,amssymb,amsthm, amsfonts}
\usepackage{booktabs}
\usepackage{graphics}
\usepackage{caption} 
\usepackage[hang]{subfigure}
\usepackage[pdftex]{graphicx}
\usepackage[czech]{babel}
\usepackage[utf8]{inputenc}
\usepackage[usenames,dvipsnames]{color}
\usepackage{url}
\usepackage{tikz}
\usepackage{amsthm}
\usepackage{float}
\usepackage{algorithm}
\usepackage{algpseudocode}
\usepackage{hyperref}
\usepackage{listings}
\usepackage{comment}
\usetikzlibrary{calc,trees,positioning,arrows,chains,shapes.geometric,%
    decorations.pathreplacing,decorations.pathmorphing,shapes,%
    matrix,shapes.symbols, decorations.markings}

\floatname{algorithm}{Pseudokód}
\renewcommand{\algorithmicrequire}{\textbf{Vstup:}}
\renewcommand{\algorithmicensure}{\textbf{Výstup:}}
\renewcommand{\lstlistingname}{Zdrojový kód}

\theoremstyle{plain}
\newtheorem{thm}{Theorem}[chapter] % reset theorem numbering for each chapter

\theoremstyle{definition}
\newtheorem{defn}[thm]{Definice} % definition numbers are dependent on theorem numbers
\newtheorem{exmp}[thm]{Example} % same for example numbers

\definecolor{src-bck}{RGB}{240, 240, 240}
\hypersetup{
    bookmarks=true,         % show bookmarks bar?
    unicode=true,           % non-Latin characters in Acrobat bookmarks
    pdftoolbar=true,        % show Acrobat toolbar?
    pdfmenubar=true,        % show Acrobat menu?
    pdffitwindow=false,     % window fit to page when opened
    pdfstartview={FitH},    % fits the width of the page to the window
    pdftitle={Bakalářská práce},    
                            % title
    pdfauthor={Jan Papoušek},% author
    pdfsubject={Paralelizace metod pro analýzu dynamických systémů pomocí grafické karty},
                            % subject of the document    
    pdfkeywords={keyword1} {key2} {key3}, % list of keywords
    pdfnewwindow=true,      % links in new window
    colorlinks=false,       % false: boxed links; true: colored links
    linkcolor=blue,         % color of internal links
    citecolor=green,        % color of links to bibliography
    filecolor=magenta,      % color of file links
    urlcolor=cyan,          % color of external links
    linkbordercolor={1 1 0}
}

\thesislang{cs}
\thesistitle{Analýza robustnosti spojitých dynamických systémů v~distribuovaném prostředí}
\thesissubtitle{Diplomová práce}
\thesisstudent{Jan Papoušek}
\thesiswoman{false}
\thesisfaculty{fi}
\thesisyear{Jaro 2013}
\thesisadvisor{RNDr. David Šafránek, Ph.D.}

\setcounter{tocdepth}{2}


\definecolor{javared}{rgb}{0.6,0,0} % for strings
\definecolor{javagreen}{rgb}{0.25,0.5,0.35} % comments
\definecolor{javapurple}{rgb}{0.5,0,0.35} % keywords
\definecolor{javadocblue}{rgb}{0.25,0.35,0.75} % javadoc
\definecolor{sourcebackground}{RGB}{250, 250, 250} % javadoc

\lstdefinestyle{Java}{
	language=Java,
	basicstyle=\ttfamily,
	keywordstyle=\color{javapurple}\bfseries,
	stringstyle=\color{javared},
	commentstyle=\color{javagreen},
	morecomment=[s][\color{javadocblue}]{/**}{*/},
	numbers=left,
	frame=leftline,
	backgroundcolor=\color{sourcebackground},
	numberstyle=\tiny\color{black},
	stepnumber=1,
	tabsize=4,
	showspaces=false,
	showstringspaces=false,
	captionpos=b
}

\lstdefinestyle{Bash}{
	language=Bash,
	basicstyle=\ttfamily,
	numbers=left,
	frame=leftline,
	numberstyle=\tiny\color{black},
	stepnumber=1,
	tabsize=4,
	showspaces=false,
	showstringspaces=false,
	captionpos=b
}

\begin{document}

\FrontMatter
\ThesisTitlePage

\begin{ThesisDeclaration}
\DeclarationText

\medskip
\begin{flushright}
\begin{minipage}{0.36\textwidth}
V Brně dne \today\\
Jan Papoušek
\end{minipage}
\end{flushright}

\AdvisorName
\end{ThesisDeclaration}

\begin{ThesisThanks}
Děkuji vedoucímu své práce Davidu Šafránkovi za odborné vedení
a~pos\-kyt\-nu\-tí cenných rad, Svenu Dražanovi za inspirativní konzultace, Tomáši Vejpustkovi za spolupráci 
při implementaci nástroje Parasim a celé Laboratoři sys\-té\-mo\-vé biologie za poskytnutí
technického zázemí.

\vspace{0.5cm}
\noindent
Děkuji svým rodičum Janu a Evě Papouškovým za podporu, kterou mi poskytli během studia a vůbec v celém
mém dosavadním životě. Děkuji své nastávající manželce Tereze Doležalové za naše dlouhé
rozpravy formující mé životní směřování, jenž vy\-ústi\-lo nejen v tuto práci.

\vspace{0.5cm}
\noindent
Bez Vás by tato práce nevznikla.

\end{ThesisThanks}

\begin{ThesisAbstract}
Práce prezentuje algoritmus pro analýzu dynamických systémů zadaných pomocí soustavy
obyčejných diferenciálních rovnic vzhledem k~cho\-vání charakterizovanému vlastností
temporální logiky signálů. Výsled\-kem a\-na\-lý\-zy je představa o tom,
jakým způsobem ovlivňují změny modelu jeho chování. Popisovaná metoda se zakládá
na již existujícím algoritmu a výpoč\-tu lokální robustnosti.

Algoritmus byl implementován v programovacím jazyce Java do podoby nástroje Parasim tak,
aby analýzu bylo možno spustit v prostředí se sdílenou nebo distribuovanou pamětí.
Výpočetní model a architektura nástroje umožňují komponenty odpovídající jednotlivým
částem algoritmu snadno nahrazovat, případně použít tuto platformu pro jiný typ
vý\-poč\-tu.

Vlastnosti algoritmu a škálovatelnost implementace pro sdílenou i distribuovanou paměť
byly ověřeny spuštěním analýzy nad vybranými modely.


\begin{comment}
The work presents an algorithm for the analysis of dynamic systems specified by the ordinary differential equations.
The system behavior is characterized by the formula of signal temporal logic. The result of the analysis is the idea of
how the changes affect the model's behavior. The described method is based on an existing algorithm and calculation of local robustnes.

The algorithm has been implemented in the Java programming language into the tool called Parasim
so that the analysis could be executed in an environment with shared or distributed memory.
Computational model and tool architecture allow components corresponding to parts of the algorithm
easy to replace, or use this platform to another type of calculation.

Properties of the algorithm implementation and scalability for both shared and
distributed memory has been verified by running analyzes on selected models.
\end{comment}

\end{ThesisAbstract}

\begin{ThesisKeyWords}
dynamický systém, soustava diferenciálních rovnic, STL, monitoring, analýza vlastnosti, robustnost
\end{ThesisKeyWords}

\MainMatter
\tableofcontents

%------------------------------- CHAPTERS --------------------------------------

\include{chapter-introduction}
\include{chapter-preliminaries}
\include{chapter-algorithm}
\include{chapter-implementation}
\include{chapter-evaluation}
\include{chapter-conclusion}

\bibliographystyle{czechiso}
\bibliography{literature}

\appendix
\include{appendix-usage}
\include{appendix-extension}
\include{appendix-configuration}
\include{appendix-extensions}
\include{appendix-measurement}

\end{document}

